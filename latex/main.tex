\documentclass[11pt,letterpaper]{article}
\usepackage[utf8]{inputenc}
\usepackage[left=1in,right=1in,top=1in,bottom=1in]{geometry}
% -----------------------------------
\usepackage{hyperref}
\hypersetup{%
  colorlinks=true,
  linkcolor=blue,
  citecolor=blue,
  urlcolor=blue,
  linkbordercolor={0 0 1}
}
% -----------------------------------
\usepackage[authordate,backend=biber]{biblatex-chicago}
\addbibresource{citation.bib}
% -----------------------------------
\title{Zotero Tutorial: \LaTeX\ Examples}
\author{Ryan Sh\`iji\'e D\`u}
\date{\today}
% -----------------------------------
\setlength{\parindent}{0.0in}
\setlength{\parskip}{0.1in}
% -----------------------------------
\begin{document}
\newcommand{\de}{\mathrm{d}}
\newcommand{\DD}{\mathrm{D}}
\newcommand{\pe}{\partial}
\newcommand{\mcal}{\mathcal}
%\newcommand{\pdx}{\left|\frac{\partial}{\partial_x}\right|}

\newcommand{\dsp}{\displaystyle}

\newcommand{\norm}[1]{\left\Vert #1 \right\Vert}
\newcommand{\mean}[1]{\left\langle #1 \right\rangle}
\newcommand{\inner}[2]{\left\langle #1,#2\right\rangle}

\newcommand{\ve}[1]{\boldsymbol{#1}}

\newcommand{\thus}{\Rightarrow \quad }
\newcommand{\fff}{\iff\quad}
\newcommand{\qdt}[1]{\quad \mbox{#1} \quad}

\newcommand{\re}{\mathrm{Re}}
\newcommand{\im}{\mathrm{Im}}
\newcommand{\E}{\mathbb{E}}
\newcommand{\lap} {\nabla^2}
\renewcommand{\div}{\nabla\cdot}

\newcommand{\hot}{\text{h.o.t.}}

\newcommand{\ssp}{\left.\qquad\right.}

\newcommand{\var}{\text{var}}
\newcommand{\cov}{\text{cov}}




\maketitle

With 
\href{https://retorque.re/zotero-better-bibtex/}{Better BibTeX}
and 
\href{https://github.com/wshanks/Zutilo}{Zutilo}, we can obtain the correct \texttt{.bib} entries, \verb|\cite{...}| command, and the formatted citation from simple shortcuts. 

For example, I could cite these entries about the tears of wine \parencite{DuklerEtAl_20, PhysicsGirl_19}. The first is a journal article and the second is a Youtube video. If one wants to share this paper with collaborator via email say, then it would be convenient to be able to obtain a formatted citation quickly. This is possible as well:\vspace*{2mm}\\
\;\hspace*{1cm}
\begin{minipage}{.85\textwidth}
    Dukler, Y., Ji, H., Falcon, C., Bertozzi, A.L., 2020. Theory for undercompressive shocks in tears of wine. Physical Review Fluids 5, 34002. https://doi.org/10.1103/PhysRevFluids.5.034002
\end{minipage}

Our ``pipeline'' preserves special characters. For example, the \"o in the title of this paper is printed in the References section correctly \parencite{Vanneste_21}; as well as the accents in the author names of this paper \parencite{Caspar-CohenEtAl_21}.

Let's cite some things others than journal articles. \cite{Diamantakis_14} is a conference paper, \cite{Buhler_14b} is a book chapter from the book \cite{Buhler_14}, and \cite{TERANEtAl_18} is a patent. Here we cite the thesis and a presentation by the same author \parencite{Shakespeare_15,Shakespeare_21}. The citation for the presentation could be better. This is likely because presentation is categorized as \verb|@misc| in the \texttt{.bib} file. 

Automation is great, but we should still be careful about the details. For example, this data is not cited well (e.g.: \cite{cisl_rda_ds633.3}). The recommended citation for this data product is:\vspace*{2mm}\\
\;\hspace*{1cm}
\begin{minipage}{.85\textwidth}%
European Centre for Medium-Range Weather Forecasts (2019): ERA5 Reanalysis (0.25 Degree Latitude-Longitude Grid). Research Data Archive at the National Center for Atmospheric Research, Computational and Information Systems Laboratory. Dataset. https://doi.org/10.5065/BH6N-5N20. Accessed† dd mmm yyyy.
\end{minipage}%

\section{Live Demonstration}
\cite{ChienCramer_22, MaxianDonev_22}

Chien, S.Y., Cramer, M.S., 2022. Compressible high-pressure lubrication flows in thrust bearings. Journal of Fluid Mechanics 939. https://doi.org/10.1017/jfm.2022.240

Maxian, O., Donev, A., 2022. Slender body theories for rotating filaments. arXiv:2203.12059 [math-ph, physics:physics].




\vfill
\printbibliography


\end{document}
